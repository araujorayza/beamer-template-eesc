%%%%%%%%%%%%%%%%%%%%%%%%%%%%%%%% FRAME %%%%%%%%%%%%%%%%%%%%%%%%%%%%%%%%

\begin{frame}{Quadcopter Model}
\begin{block}
	
Considering the the quadrotor's response to the input command vector  $u = \begin{bmatrix}
u_{\nu_x}&
u_{\nu_y}&
u_{\dot{z}}&
u_{\dot{\psi}}
\end{bmatrix}^T$, the dynamics can be modeled as the following linear system \autocite{VagoSantana2014a}

\begin{equation} \label{eq:Sant_lin}
\ddot{q}_b=\Gamma_A\dot{q}_b+\Gamma_Bu
\end{equation}
where $q_b = \begin{bmatrix}
x_b& 
{y}_b&
{z}_b&
{\psi}_b
\end{bmatrix}^T$ \begin{equation}
\Gamma_A=-\begin{bmatrix}
\gamma_2 & & & \\
&\gamma_4 & &\\
& &\gamma_6 &\\
& & &\gamma_8
\end{bmatrix}, ~\Gamma_B=\begin{bmatrix}
\gamma_1 & & & \\
&\gamma_3 & &\\
& &\gamma_5 &\\
& & &\gamma_7
\end{bmatrix}
\end{equation}

\end{block}

\end{frame}

%%%%%%%%%%%%%%%%%%%%%%%%%%%%%%%% FRAME %%%%%%%%%%%%%%%%%%%%%%%%%%%%%%%%
\begin{frame}
	For control purposes, it's more convenient to work with position and orientation with respect to a global stationary frame. \newline
	\newline
	Considering rotation only in the $z$ axis, we have
	\begin{equation} \label{eq:vel_trans}
	% ^{W}\dot{q} &= \prescript{W}{B}R ^{B}\dot{q}
	\dot{q} = R\dot{q_b} 
	\end{equation}
	where
	\begin{equation} \label{eq:R_matrix}
	R = \begin{bmatrix}
	\cos{\psi} & -\sin{\psi} & 0 & 0 \\
	\sin{\psi} & \cos{\psi} & 0 & 0 \\
	0 & 0 & 1 & 0 \\
	0 & 0 & 0 & 1
	\end{bmatrix}
	\end{equation}
\end{frame}

%%%%%%%%%%%%%%%%%%%%%%%%%%%%%%%% FRAME %%%%%%%%%%%%%%%%%%%%%%%%%%%%%%%%
\begin{frame}
	\begin{block}{Santana et. al. (2014) - Global Reference Frame}
			\begin{equation} \label{eq:robotic_dyn}
			\ddot{q}+NR^{T}\dot{q} = Mu, 
			\end{equation}
			\begin{eqnarray}
			M = \begin{bmatrix}
			\gamma_1\cos{\psi} & -\gamma_3\sin{\psi} & 0 & 0 \\
			\gamma_1\sin{\psi} & \gamma_3\cos{\psi} & 0 & 0 \\
			0 & 0 & \gamma_5 & 0 \\
			0 & 0 & 0 & \gamma_7
			\end{bmatrix} \\
			N =     \begin{bmatrix}
			\gamma_2\cos{\psi} & -\gamma_4\sin{\psi} & 0 & 0 \\
			\gamma_2\sin{\psi} & \gamma_4\cos{\psi} & 0 & 0 \\
			0 & 0 & \gamma_6 & 0 \\
			0 & 0 & 0 & \gamma_8
			\end{bmatrix}.
			\end{eqnarray}
	\end{block}	
\end{frame}

%%%%%%%%%%%%%%%%%%%%%%%%%%%%%%%% FRAME %%%%%%%%%%%%%%%%%%%%%%%%%%%%%%%%
% \begin{frame}{State Space Model}
% 	Let us define a state vector $\mathbf{\zeta} = \begin{bmatrix}
% 	\dot{q}^T & 
% 	q^T
% 	\end{bmatrix}^T = \begin{bmatrix}
% 	\dot{x} & 
% 	\dot{y}&
% 	\dot{z}&
% 	\dot{\psi} &
% 	{x} & {y} & {z} & {\psi}
% 	\end{bmatrix} ^T$. After some algebraic manipulation, one can obtain
% 	\begin{equation} \label{eq:SS}
% 	\dot{\mathbf{\zeta}} =   \begin{bmatrix}
% 	-NR^T & ~0\\
% 	~I  & ~0
% 	\end{bmatrix} \mathbf{\zeta} + \begin{bmatrix}
% 	M\\
% 	0
% 	\end{bmatrix}u
% 	\end{equation}
% 	where the input command vector is given by $u = \begin{bmatrix}
% 	u_{\nu_x}&
% 	u_{\nu_y}&
% 	u_{\dot{z}}&
% 	u_{\dot{\psi}}
% 	\end{bmatrix}^T$.
% \end{frame}
%%%%%%%%%%%%%%%%%%%%%%%%%%%%%%%% FRAME %%%%%%%%%%%%%%%%%%%%%%%%%%%%%%%%
\begin{frame}{Error Model}
\begin{block}{}
	Given a twice differentiable desired state trajectory $\zeta_d(t) = \begin{bmatrix}\dot{q}_d^T & q_d^T \end{bmatrix}^T$, let us define the tracking error $e = \mathbf{\zeta} - \mathbf{\zeta}_{d}$,
	\begin{equation} \label{eq:SS_error}
\dot{e} = \begin{bmatrix} 
-NR^T & 0 \\
I & 0
\end{bmatrix}e + \begin{bmatrix}
I\\
0
\end{bmatrix}\nu
\end{equation}
where $\nu = - \ddot{q}_d -NR^T\dot{q}_d + Mu$ is a virtual control input to the error system. The control input to be sent to the quadrotor system is given by
\begin{equation} \label{eq:real_control}
u = M^{-1} (\nu + \ddot{q}_d + NR^T\dot{q}_d)
% \nu = - \ddot{q}_d -NR^T\dot{q}_d + Mu
\end{equation} 
\end{block}
\end{frame}

\begin{frame}{System Analysis}

\begin{itemize}
    \item The origin is an unstable equilibrium point for system \eqref{eq:SS_error} and has an infinite number of equilibria located in $\{ e \in \mathbb{R}^8 :  e_1, e_2, e_3, e_4 = 0 \}$
    \item Hartman-Grobman Theorem does not apply (linearized system with non hyperbolic equilibria)
    \item To make the closed loop system have an unique equilibrium point at the origin, we modify \eqref{eq:real_control} 
\begin{equation}
    u = M^{-1} \left( \nu + \ddot{q}_d + NR^T\dot{q}_d -k(q-q_d) \right),
\end{equation} \label{eq:real_cont_new}
where $k$ is a positive scalar
\item \begin{equation}
    \Rightarrow \dot{e} = \begin{bmatrix}
-NR^T & -kI\\
I & 0
\end{bmatrix}e + \begin{bmatrix}
I\\
0
\end{bmatrix} \nu \label{eq:newclosedloop}
\end{equation}
\item We have not thoroughly studied the impact of this modification, but it seems to have increased the feasibility of the various LMI conditions we tested 
\end{itemize}
    
\end{frame}


\begin{frame}{LPV system}
	The error system is given by
	\begin{equation*} \label{eq:err_simple}
	\dot{e} =  A(\psi)e + B\nu
	\end{equation*} where
	
	\begin{equation*}
			A(\psi)=-\begin{bmatrix} \begin{array}{cccc|c}
				\gamma_{2}\cos^{2}(\psi) + \gamma_{4}\sin^{2}(\psi) &  \frac{\gamma_{4}-\gamma_{2}}{2}\sin(2\psi)&  &  & \multirow{4}{*}{0}\\
				\frac{\gamma_{2}-\gamma_{4}}{2}\sin(2\psi) & \gamma_{4}\cos^{2}(\psi) + \gamma_{2}\sin^{2}(\psi) &  &  &\\
				&  & \gamma_6 & &\\
				&  &  & \gamma_8& \\ \hline
				\multicolumn{3}{c}{-I} & & 0
				\end{array}
				\end{bmatrix}
	\end{equation*} and \begin{equation*}
		B = \begin{bmatrix}
				I\\
				0
			\end{bmatrix}\nu
			\end{equation*}
\end{frame}
%%%%%%%%%%%%%%%%%%%%%%%%%%%%%%%% FRAME %%%%%%%%%%%%%%%%%%%%%%%%%%%%%%%%
\begin{frame}{Modeling alternatives}
	Different choices of parameter vectors yield different representations. For example, choosing  
	\begin{equation} \label{eq:param_init}
	\rho(t) = \begin{bmatrix}
	\gamma_{2}\cos^{2}(\psi) + \gamma_{4}\sin^{2}(\psi) \\
	\frac{\gamma_{2}-\gamma_{4}}{2}\sin(2\psi)
	\end{bmatrix}
	\end{equation} 
	as the parameter vector, one can write 
	\begin{equation} \label{eq:err_LPV}
	\dot{e} =  (A_{0} + \sum_{i=1}^{2}A_{i}\rho_{i})e + B\nu
	\end{equation} 
	for some appropriate $A_i$, $i=0,\dots, 2$. The parameters bounds are given by 
	\begin{equation} \label{eq:param_init_bounds}
	\begin{aligned}
	\rho_{1} \in [\min(\gamma_{2},\gamma_{4}), ~\max(\gamma_{2},\gamma_{4})]\\
	\rho_{2} \in [-\frac{\gamma_{2}-\gamma_{4}}{2},~\frac{\gamma_{2}-\gamma_{4}}{2}].
	\end{aligned}
	\end{equation} 
\end{frame}
%%%%%%%%%%%%%%%%%%%%%%%%%%%%%%%% FRAME %%%%%%%%%%%%%%%%%%%%%%%%%%%%%%%%
\begin{frame}{Modeling alternatives}
	As an alternative we have 
	\begin{equation} 
	\rho(t) = \begin{bmatrix}
	\cos^{2}(\psi)\\
	\sin^{2}(\psi)\\
	\sin(2\psi)
	\end{bmatrix}
	\end{equation} with bounds  
	\begin{equation} \label{eq:param_final_bounds}
	\rho_{1},\rho_{2} \in [0, 1], ~\rho_{3} \in [-1,1]
	\end{equation} 
	which gives a representation of the form
	\begin{equation} \label{eq:err_LPV_final}
	\dot{e} =  (A_{0} + \sum_{i=1}^{3}A_{i}\rho_{i})e + B\nu
	\end{equation} 
\end{frame}

\begin{frame}{Modeling alternatives}
	Using the sector nonlinearity approach, the regular choice of premises would be
	\begin{subequations}
		\begin{eqnarray}
		z_1(t)&=&	\gamma_{2}\cos^{2}(\psi) + \gamma_{4}\sin^{2}(\psi)\\
		z_2(t)&=&\frac{\gamma_{2}-\gamma_{4}}{2}\sin(2\psi)
		\end{eqnarray}
	\end{subequations}
	However, choosing simpler premises 
	\begin{subequations}
		\begin{align}
		z_1(t)=\cos{\psi(t)} \\
		z_2(t)=\sin{\psi(t)}
		\end{align}
	\end{subequations} still gives an exact representation of the error system.
\end{frame}

% \begin{frame}{TS-fuzzy modeling}
% Continuing to build the TS-fuzzy model, we have 
% \begin{subequations} 
% 	\begin{align}
% 	z_1(t)&=\cos{\psi(t)}=(1)\omega_1^1+(-1)\omega_0^1 \\
% 	z_2(t)&=\sin{\psi(t)}=(1)\omega_1^2+(-1)\omega_0^2,
% 	\end{align}
% \end{subequations} where

% \begin{equation} 
% \sum_{i=0}^{1}\omega_{i}^{k} = 1 ~\forall k \in \{1,2\}.
% \end{equation}
% \end{frame}
%%%%%%%%%%%%%%%%%%%%%%%%%%%%%%%% FRAME %%%%%%%%%%%%%%%%%%%%%%%%%%%%%%%%
% \begin{frame}{TS modeling}
% 	We, then, find \begin{subequations} 
% 		\begin{align}
% 		w_0^1=\dfrac{z_1+1}{2},~ w_1^1=1-w_0^1\\
% 		w_0^2=\dfrac{z_2+1}{2},~ w_1^2=1-w_0^2.
% 		\end{align}
% 	\end{subequations}
% \end{frame}

\begin{frame}
	\begin{block}{TS-fuzzy model}
	The tracking error system can be thus represented as
	\begin{equation}
	\dot{e}(t) = \sum_{i=1}^{4}\sum_{j=1}^{4}h_i h_jA_{ij}e+B\nu =\sum_{i=1}^{4}\sum_{j=1}^{4}h_i h_j\begin{bmatrix}
	-N_iR_j^T & 0 \\
	I & 0
	\end{bmatrix}e+\begin{bmatrix}
	I\\
	0
	\end{bmatrix}\nu
	\end{equation} where $N_i,R_j$ are the matrices of the local models of the system and 
		\begin{equation} 
		\forall\, i\in\{1,\ldots, 4\},~~h_i(\psi) \geq 0 \text{~~and~~} \sum_{i=1}^{r}h_i(\psi)=1.
		\end{equation}
		\begin{equation}
		\begin{aligned}
		h_{1}(\psi)&=w_1^1(\psi)w_1^2(\psi),& h_{2}(\psi)&=w_1^1(\psi)w_0^2(\psi),\\
		h_{3}(\psi)&=w_0^1(\psi)w_1^2(\psi),& h_{4}(\psi)&=w_0^1(\psi)w_0^2(\psi). \\
			w_0^1&=\dfrac{\cos(\psi(t))+1}{2},~& w_1^1&=1-w_0^1\\
	w_0^2&=\dfrac{\sin(\psi(t))+1}{2},~& w_1^2&=1-w_0^2.
		\end{aligned}    
		\end{equation}

	\end{block}
\end{frame}

\begin{frame}{Models summary}
    \begin{table}[]
        \centering
        \begin{tabular}{c|c}
             $\#$ & Description \\ \hline
             1& 4 subsystems \\ \hline
             2& 16 subsystems\\ \hline
             3& 10 subsystems\\ \hline
             4& 8 subsystems\\ \hline
        \end{tabular}
        \caption{Summary of different modeling alternatives we developed}
        \label{tab:my_label}
    \end{table}
    \begin{block}{}
    All the models are equivalent to each other and exactly represent the nonlinear system. For proof, check \cite{Araujo2019}
    \end{block}
\end{frame}
    % \begin{frame}{TS modeling}
    % Rewriting the TS-fuzzy representation in a quasi LPV notation, we have
    % \begin{equation}
    % \dot{e}(t) = \mathcal{A}(\rho)e+B\nu
    % \end{equation}
    % with
    % \begin{eqnarray}
    % \mathcal{A}(\rho)& =& \sum_{i=1}^{9}\rho_i\mathcal{A}_i \nonumber\\&=&  \sum_{i=1}^{4}\sum_{j=1}^{4}h_i(\psi)h_j(\psi)\begin{bmatrix}
    % -N_iR_j^T & 0 \\
    % I & 0
    % \end{bmatrix}.
    % \end{eqnarray}
    % \end{frame}


