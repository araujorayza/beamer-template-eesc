\begin{frame}{Promising Results}
	\begin{enumerate}
		\item Workflow for simulation and experimental results is successfully implemented and works well 
		\item Errors may be caused by erroneous state estimation
		\item System does not become unstable when nonlinearities are excited
	\end{enumerate}
\end{frame}


\begin{frame}{What has been accomplished}
	\begin{enumerate}
		\item Complete PhD course requirements
		\item Literature Review
		\item Description and simulation of the drone model\autocite{VagoSantana2014a}
		\item TS-fuzzy modelling via sector nonlinearity and model validation
	\end{enumerate}
\end{frame}

\begin{frame}{What has been accomplished}
	\begin{enumerate}
		\setcounter{enumi}{4}
		\item Tracking problem formulation and error modeling
		\item TS-Fuzzy modelling of error system and model validation
		\item Design and simulation of control strategy
		\item Experimental environment setup
		\item Controller implementation in C++
		\item Testing: Parrot-Sphinx simulation and indoor experiments
	\end{enumerate}
\end{frame}

%\begin{frame}{Goal Checklist}
%\begin{CheckList}{Goal}
%	\Goal{achieved}{Overview the LPV framework*}
%	\Goal{achieved}{Identify the similarities between the LPV and TS-fuzzy frameworks*}
%	\Goal{achieved}{Design a control system for a quadrotor to track a desired trajectory using a gain scheduling controller}
%	\Goal{achieved}{Implement the designed control system, run simulations and experiments to evaluate the controller performance}
%\end{CheckList}
%\end{frame}

\begin{frame}{Next Research Steps}
	\begin{enumerate}
		\item Implementing trajectory with varying $\psi$
		
		\item Adding polytopic model uncertainties
		
		\item Changing velocity model in Kalman Filter
		
    	\item Adding state estimation to the problem formulation and stability conditions
	\end{enumerate}
	\end{frame}

%%%%%%%%%%%%%%%%%%%%%%%%%%%%%%%% FRAME %%%%%%%%%%%%%%%%%%%%%%%%%%%%%%%%
\begin{frame}{Next Research Steps}
	\begin{enumerate} 
			\setcounter{enumi}{4}
    	\item Studying different techniques to reduce conservatism of control design
    	\begin{itemize} \vspace{-0.45cm}
    		\item Exploring membership function's properties
    		\item Using the extension of LaSalle's Invariance Principle
    	\end{itemize}
    	
    	\item Reformulating the tracking problem as a static output feedback control

	\end{enumerate}
\end{frame}


%%%%%%%%%%%%%%%%%%%%%%%%%%%%%%%% FRAME %%%%%%%%%%%%%%%%%%%%%%%%%%%%%%%%
\begin{frame}{Timeline}
%\begin{table}[!htp]
%	\setlength{\tabcolsep}{4pt}
%	%\renewcommand{\arraystretch}{1.5}
%	\centering
%	\begin{tabular}{|c|c|c|c|c|c|}%>{\centering}p{8mm}|>{\centering}p{8mm}|>{\centering}p{8mm}|>{\centering}p{8mm}|>{\centering}p{8mm}|>{\centering}p{8mm}|}
%		\hline
%		\raisebox{0pt}[12pt][6pt]{} & \multicolumn{1}{|c|}{1/2019} & \multicolumn{1}{|c|}{2/2019} & \multicolumn{1}{|c|}{1/2020} & \multicolumn{1}{|c|}{2/2020} & \multicolumn{1}{|c|}{1/2021} \\
%		\hline
%		\raisebox{0pt}[12pt][6pt]{Item 1}   & x & x &  & &  \\
%		\hline
%		\raisebox{0pt}[12pt][6pt]{Item 2.a} & x & x &  & &\\
%		\hline
%		\raisebox{0pt}[12pt][6pt]{Item 2.b} & x &  &  & & \\
%		\hline
%		\raisebox{0pt}[12pt][6pt]{Item 2.c} &  & x &  & & \\
%		\hline
%		\raisebox{0pt}[12pt][6pt]{Item 3}   & x & x &  & & \\
%		\hline
%		\raisebox{0pt}[12pt][6pt]{Item 4}   &  & x & x & & \\
%		\hline
%		\raisebox{0pt}[12pt][6pt]{Item 5}   &  & x & x & x & \\
%		\hline
%		\raisebox{0pt}[12pt][6pt]{Item 6}   &  &   &  x & x & \\
%		\hline
%		\raisebox{0pt}[12pt][6pt]{Item 7}   &  &   &   & x & x \\
%		\hline
%		\raisebox{0pt}[12pt][6pt]{Item 8}   & x & x & x & x & x \\
%		\hline
%	\end{tabular}
%\end{table}
\begin{table}[!htp]
	\caption{PhD timeline from october 2019 to july 2021}\label{crono}
	\setlength{\tabcolsep}{4pt}
	%\renewcommand{\arraystretch}{1.5}
	\centering
	\begin{tabular}{||c|c|c|c|c||}%>{\centering}p{8mm}|>{\centering}p{8mm}|>{\centering}p{8mm}|>{\centering}p{8mm}|>{\centering}p{8mm}|>{\centering}p{8mm}|}
		\hline
		\raisebox{0pt}[12pt][6pt]{} & 
		\multicolumn{1}{|c|}{5$^{th}$  sem}  &
		\multicolumn{1}{|c|}{6$^{th}$  sem}  &
		\multicolumn{1}{|c|}{7$^{th}$  sem}  &
		\multicolumn{1}{|c||}{8$^{th}$  sem} \\
		\hline \hline
		\raisebox{0pt}[12pt][6pt]{Item 1}   &  x   	& x &  & \\
		\hline
		\raisebox{0pt}[12pt][6pt]{Item 2}   &  x 	&  	&  &\\
		\hline
		\raisebox{0pt}[12pt][6pt]{Item 3}   &  x    & x &  & \\
		\hline
		\raisebox{0pt}[12pt][6pt]{Item 4}   &  x    & x &  & \\
		\hline
		\raisebox{0pt}[12pt][6pt]{Item 5}   &      	& x & x & \\
		\hline
	\end{tabular}
\end{table}
\end{frame}
%%%%%%%%%%%%%%%%%%%%%%%%%%%%%%%% FRAME %%%%%%%%%%%%%%%%%%%%%%%%%%%%%%%%

\begin{frame}{Published or Submitted Papers}
	\begin{itemize}
		\item R. F. Q. Magossi, R. A. Bezerra, P. V Leme, and V. A. Oliveira, “PID controller design based on $\mathcal{H}_\infty$ performance,” in XIII Simpósio Brasileiro de Automação Inteligente, 2017.
		
		\item F. A. Faria, L. J. Elias, R. Araujo, and V. A. Oliveira, “Less conservative state feedback design conditions for switched Takagi-Sugeno fuzzy systems,” in 2019 18th European Control Conference (ECC), 2019, pp. 3698–3703.
	\end{itemize}

\end{frame}



\begin{frame}{Published or Submitted Papers}
	\begin{itemize}
		
		\item R. Araujo, Flávio Faria, Leandro J. Elias, Vilma A. Oliveira, "Trajectory Tracking for the Bebop Parrot quadrotor using Takagi-Sugeno fuzzy models", in XIV Simpósio Brasileiro de Automação Inteligente, 2019. \textit{Accepted}
		
		\item R. Araujo, Leandro J. Elias, Flávio Faria, Vilma A. Oliveira, "On the selection of membership functions of TS fuzzy models for a commercial quadrotor", in XIV Conferência Brasileira de Dinâmica, Controle e Aplicações, 2019, \textit{Submitted}.
		
		
		\item L. J. Elias, F. A. Faria, R. Araujo, and V. A. Oliveira, “Less conservative stabilizing conditions for Takagi-Sugeno systems using a switched fuzzy Lyapunov function,” IEEE Trans. Fuzzy Syst., 2019, \textit{Submitted}.
		
	\end{itemize}
	
\end{frame}



%%%%%%%%%%%%%%%%%%%%%%%%%%%%%%%% FRAME %%%%%%%%%%%%%%%%%%%%%%%%%%%%%%%%

