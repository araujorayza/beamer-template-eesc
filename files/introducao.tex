\begin{frame}{Motivation}
\begin{figure}
\centering
\includegraphics[width=.8\textwidth]{figuras/images}
\caption{Intel light show on July 15, 2018 Source: \textit{VCG/VCG via Getty Images} (2018)}
% \label{fig:my_label}
\end{figure}
\end{frame}

%%%%%%%%%%%%%%%%%%%%%%%%%%%%%%  FRAME %%%%%%%%%%%%%%%%%%%%%%%%%%%%%%

\begin{frame}{Motivation}
\begin{itemize}
	\item Nonlinear systems are ubiquitous in real applications
	\item Linear controllers are not ideal and not always possible
	\item Gain-scheduling is a suiting middle ground
% 	\item Feedback Linearization \autocite{VagoSantana2014a} 
% 	\item Linear  Quadratic Regulator + Sliding mode \autocite{Tang2015} 
\end{itemize}
\end{frame}

%%%%%%%%%%%%%%%%%%%%%%%%%%%%%%  FRAME %%%%%%%%%%%%%%%%%%%%%%%%%%%%%%


\begin{frame}{Motivation}{Gain-scheduling design}
\begin{columns}
\begin{column}{0.45\textwidth}
\vspace{-0.10cm}
\begin{block}{Linear Parameter Varying Framework}
    \begin{itemize}
       \item Linear systems with parameters that vary with time during operation
       \item Allows the systematic design of gain scheduling controllers
   \end{itemize}
\end{block}
\end{column}
\begin{column}{0.5\textwidth}  %%<--- here
	\vspace{0.50cm}
    \begin{itemize}
    	\item Design conditions: Lyapunov theory, the small-gain theorem or invariant set theory.
    	\item Research focused on reducing conservatism
    \end{itemize}
\end{column}
\end{columns}
\end{frame}
%%%%%%%%%%%%%%%%%%%%%%%%%%%%%%  FRAME %%%%%%%%%%%%%%%%%%%%%%%%%%%%%%

\begin{frame}{Motivation}{}
	    \begin{itemize}
	    	\item LPV vs Takagi-Sugeno Fuzzy 
	    	\item The TS-fuzzy framework can represent nonlinear systems exactly
	    	\item Design conditions are LMIs
	    	\item Most applications in the literature are simulations
	    \end{itemize}
\end{frame}

%%%%%%%%%%%%%%%%%%%%%%%%%%%%%%  FRAME %%%%%%%%%%%%%%%%%%%%%%%%%%%%%%

\begin{frame}{Objectives}


\begin{block}{Main Objectives}
\begin{itemize}
	\item Overview  the LPV framework;
	\item Identify the similarities between the LPV and TS-fuzzy frameworks;
	\item Design a control system for a quadrotor to track a desired trajectory using a gain scheduling controller.  
	\item Implement the designed control system, run simulations and experiments to evaluate the controller performance.
\end{itemize}
\end{block}
\end{frame}